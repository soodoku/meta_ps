\documentclass[11pt]{article}
\usepackage{color}
\usepackage[usenames,dvipsnames,svgnames,table]{xcolor}
\definecolor{dark-red}{rgb}{0.7,0.1,0.1} 
\definecolor{dark-blue}{rgb}{0,0,0.7} 
\usepackage[linkcolor=dark-red,
            colorlinks=true,
            urlcolor=dark-blue,
            pdfstartview={XYZ null null 1.00},
            pdfauthor={Gaurav Sood, gsood07@gmail.com},
            citecolor=dark-red,
            bookmarks=false,
            pdfborder={0 0 0},
            pdftitle={The Review}]{hyperref}
            
\usepackage{amsfonts,amssymb,amsbsy,amsmath,amsxtra}

\usepackage{indentfirst}
\usepackage{setspace} % To set line spacing
\usepackage{multirow}

\usepackage{verbatim}

\usepackage[multiple]{footmisc}
\usepackage{fancyvrb}

\usepackage{longtable}

\usepackage[margin=1in]{geometry}
\usepackage{graphicx}

\raggedright
\parindent=1.5em % <- or whatever indent you want

\usepackage{natbib}
\usepackage{url}
\begin{comment}

setwd(paste0(githubdir, "/meta_apsr/ms/"))	
tools::texi2dvi("the_review.tex", pdf=TRUE, clean=TRUE)	
setwd(paste0(githubdir, "/meta_apsr/"))

\end{comment}

\begin{document}
\title{\vspace{-1cm}\normalsize{\textbf{The Review: Production and Consumption of APSR Articles}}}
\author{Gaurav Sood\\\small{\href{mailto:gsood07@gmail.com}{\tt{gsood07@gmail.com}}}}
\maketitle
\begin{center}
\vspace{-.5cm}\textbf{NB:} Preliminary draft. Please do not cite without permission.
\end{center}
\vspace{.2cm}
\doublespacing

Scientific production is affected by a variety of variables that have little to do with the science. Availability of funding, fads, various parts of the scientific paper production pipeline: (artificial) limits on journal space, the number of articles editors must review, the incentives to be nice to the author, etc., among other things, affects what topic is researched, the quality of research, and who produces and reads it, among other things. Using data on all articles published in the The American Political Science Review, the pre-eminent journal in political science since its inception nearly a hundred years ago, I shed light on some of those issues in political science.\footnote{The data and scripts to acquire, process and analyze the data can be downloaded from \href{https://github.com/soodoku/meta_apsr}{https://github.com/soodoku/meta\_apsr}.} 

Over the past 100 or so years, article length has shown marked variability (see Figure~\ref{fig:pages}). There is a marked see-saw pattern in the average length of the article, but unlike top economics journals we don't see a marked trend towards longer articles \citep{card2013nine, card2014page}. (It is very likely, however, that the length of online appendices has grown substantially.)

\begin{figure}[htbp]
\centering
\caption{Number of Pages per Article Over Time}
\includegraphics[scale=.85]{../figs/n_pages_per_article_over_time.pdf}
\label{fig:pages}
\end{figure}

Over time, the number of articles per issue has shown a sharp increase --- over 100 years, the number of articles has more than doubled (see Figure~\ref{fig:narticles}). Though, over the past thirty or so years, the number of articles per issue has remained steady. 

\begin{figure}[htbp]
\centering
\caption{Number of Articles per Issue Over Time}
\includegraphics[scale=.85]{../figs/articles_per_issue_over_time.pdf}
\label{fig:narticles}
\end{figure}

Given the two things we find above, it is obvious that pages per issue would have increased. And so we find (see Figure~\ref{fig:issue}). 

\begin{figure}[htbp]
\centering
\caption{Pages per Issue Over time}
\includegraphics[scale=.85]{../figs/pages_per_issue_over_time.pdf}
\label{fig:issue}
\end{figure}

Looking at production, I tallied two features: number of authors per article over time, and proportion of female authors per article over time. Like with other sciences, co-authorship is on the rise. Though, solo authored papers still make a sizable proportion of publications in the APSR, the modal publication today has two authors (see Figure~\ref{fig:nauthors}).

\begin{figure}[htbp]
\centering
\caption{Number of Authors per Article Over Time}
\includegraphics[scale=.85]{../figs/pages_per_issue_over_time.pdf}
\label{fig:nauthors}
\end{figure}

The data on proportion of women authors per article is distressing. While again the proportion of women on each article published in the APSR has been rising, the average article still has just 20\% female authors (see Figure~\ref{fig:women}).  

\begin{figure}[htbp]
\centering
\caption{Percentage of Matches Won Minus Matches Lost After Winning the Toss by Difference in Ranks}
\includegraphics[scale=.85]{../figs/gender_authors_per_article_over_time.pdf}
\label{fig:women}
\end{figure}

Flipping the lens and looking at two consumption metrics: number of abstract views and number of full-text views, provides a familiar power-law distribution. Most of the articles (abstracts) aren't viewed at all. And a small set of articles gets a lot of views (see Figures~\ref{fig:abstracts} and ~\ref{fig:fulltext}). .

\begin{figure}[htbp]
\centering
\caption{Abstract Views}
\includegraphics[scale=.85]{../figs/abstract_views.pdf}
\label{fig:abstracts}
\end{figure}

\begin{figure}[htbp]
\centering
\caption{Full-Text Views}
\includegraphics[scale=.85]{../figs/fulltext_views.pdf}
\label{fig:fulltext}
\end{figure}

\clearpage
\bibliographystyle{apsr}
\bibliography{reviewbib}

\end{document}